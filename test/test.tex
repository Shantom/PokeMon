\documentclass{article}
\usepackage{xeCJK}
\usepackage{graphics}
\usepackage[CJKbookmarks=true,colorlinks,linkcolor=black,anchorcolor=blue,citecolor=green]{hyperref}
\usepackage{float}
\usepackage{multirow}



\begin{document}
\title{宠物小精灵在线游戏系统测试报告}
\author{项目组成员:史柠玮}
\date{\today}


\maketitle
\pagebreak[4]
\tableofcontents
\pagebreak[4]


\section{范围}

\subsection{项目概述}
\begin{enumerate}
\item 被测软件名称\\
  宠物小精灵在线游戏系统
\item 版本\\V1.0
\item 被测软件用途\\
  适合作为无聊时消遣
\item 被测软件的组成
  \begin{itemize}
  \item 服务器
    \begin{itemize}
    \item 与客户端通信模块
    \item 与数据库交互模块
    \end{itemize}
  \item 客户端
    \begin{itemize}
    \item 注册登录模块
    \item 主界面
    \item 查看所有用户及其精灵
    \item 战斗模块
    \end{itemize}
  \end{itemize}
\item 被测软件功能
  \begin{enumerate}
  \item 客户端注册以及登录
  \item 客户端查看所有用户是否在线以及拥有的精灵信息
  \item 客户端进行升级战以及决斗战
  \item 服务端查看所有用户是否在线以及拥有的精灵信息
  \end{enumerate}
\end{enumerate}

\subsection{测试概述}

\subsubsection{测试对象概述}

\begin{table}[H]
  \centering
  \caption{被测对象一览表}
  \begin{tabular}{|c|c|c|c|}
    \hline
    序号&\multicolumn{2}{|c|}{被测对象}&测试场地\\ \hline
    \multirow{5}*{TEST-1}& \multirow{5}*{客户端功能测试}&注册&\multirow{5}*{真实环境}\\ 
        &&登录&\\
        &&查看全部用户及其精灵&\\
        &&升级战&\\
        &&决斗战&\\ \hline
    \multirow{2}*{TEST-2}& \multirow{2}*{服务端功能测试}&查看用户登录登出情况&\multirow{2}*{真实环境}\\
    &&查看所有用户及其精灵&\\ \hline
  \end{tabular}
\end{table}

\section{测试结果}

\subsection{客户端功能测试(TEST-1)}

\subsubsection{客户端-注册}
\begin{itemize}
\item 测试用例名称:客户端-注册
\item 标识:TEST-1.1
\item 测试用例综述:检验用户注册正确性。
\item 用例初始化:无
\item 前提和约束:无
\item 测试步骤
  \begin{enumerate}
  \item
    \begin{itemize}
    \item 操作:输入任意正确的帐号(test)密码(1234qwer)与重复密码(1234qwer)
    \item 结果和评估标准:提示注册成功
    \item 实测结果:与预期结果一致
    \end{itemize}
  \item
    \begin{itemize}
    \item 操作:输入刚刚注册过的帐号(test)与任意密码(1234qwer)与重复密码(1234qwer)
    \item 结果和评估标准:提示用户名已存在
    \item 实测结果:与预期结果一致
    \end{itemize}
  \item
    \begin{itemize}
    \item 操作:输入任意帐号(test)与任意密码(1234qwer)与不同的重复密码(1234qwe)
    \item 结果和评估标准:提示两次密码不一致
    \item 实测结果:与预期结果一致
    \end{itemize}
  \end{enumerate}
\item 测试用例终止条件:本测试用例的全部测试步骤被执行或因某种原因导致测试步骤无法执行(异常终止)。
\item 测试用例通过准则:本测试用例的全部测试步骤都通过即标志本用例为 “通过”。
\item 设计人员:史柠玮
\item 预计执行时长:5min
\item 执行状态:完成
\item 执行结果:通过
\item 问题单标识:---
\item 测试人员:史柠玮
\item 测试时间: \today
\item 未执行原因:---
\end{itemize}

\subsubsection{客户端-登录}
\begin{itemize}
\item 测试用例名称:客户端-登录
\item 标识:TEST-1.2
\item 测试用例综述:检验用户登录正确性。
\item 用例初始化:已注册过的一个帐号(test),密码(1234qwer)
\item 前提和约束:已注册过的一个帐号(test),密码(1234qwer)
\item 测试步骤
  \begin{enumerate}
  \item
    \begin{itemize}
    \item 操作:输入正确的帐号(test),密码(1234qwer)
    \item 结果和评估标准:提示登录成功,并进入主界面
    \item 实测结果:与预期结果一致
    \end{itemize}
  \item
    \begin{itemize}
    \item 操作:输入正确的帐号(test),错误的密码(1234qwe)
    \item 结果和评估标准:提示用户名或密码错误
    \item 实测结果:与预期结果一致
    \end{itemize}
  \item
    \begin{itemize}
    \item 操作:输入错误的帐号(test1),任意密码(1234qwer)
    \item 结果和评估标准:提示用户名或密码错误
    \item 实测结果:与预期结果一致
    \end{itemize}
  \end{enumerate}
  
\item 测试用例终止条件:本测试用例的全部测试步骤被执行或因某种原因导致测试步骤无法执行(异常终止)。
\item 测试用例通过准则:本测试用例的全部测试步骤都通过即标志本用例为 “通过”。
\item 设计人员:史柠玮
\item 预计执行时长:5min
\item 执行状态:完成
\item 执行结果:通过
\item 问题单标识:---
\item 测试人员:史柠玮
\item 测试时间: \today
\item 未执行原因:---
\end{itemize}

\subsubsection{客户端-查看全部用户及其精灵}
\begin{itemize}
\item 测试用例名称:客户端-查看全部用户及其精灵
\item 标识:TEST-1.3
\item 测试用例综述:测试客户端主界面上的All Users按钮以及打开界面的功能
\item 用例初始化:已经成功登录
\item 前提和约束:打开多个客户端并用不同帐号成功登录
\item 测试步骤
  \begin{enumerate}
  \item
    \begin{itemize}
    \item 操作:在一个成功登录的客户端主界面点击All Users按钮
    \item 结果和评估标准:成功打开All Users窗口,里面显示了所有注册用户,其中在线用户字体颜色为白色,离线用户字体颜色
      为灰色,右侧默认显示第一个用户的精灵信息
    \item 实测结果:与预期结果一致
    \end{itemize}
  \item
    \begin{itemize}
    \item 操作:在All Users 窗口左侧列表中点击某一用户(test),并点击show PokeMon按钮
    \item 结果和评估标准:在右侧列表中显示出了该用户的精灵信息
    \item 实测结果:与预期结果一致
    \end{itemize}

  \item
    \begin{itemize}
    \item 操作:在All Users 窗口左侧列表中\emph{双击}某一用户(test)
    \item 结果和评估标准:在右侧列表中显示出了该用户的精灵信息
    \item 实测结果:与预期结果一致
    \end{itemize}



  \end{enumerate}
  
\item 测试用例终止条件:本测试用例的全部测试步骤被执行或因某种原因导致测试步骤无法执行(异常终止)。
\item 测试用例通过准则:本测试用例的全部测试步骤都通过即标志本用例为 “通过”。
\item 设计人员:史柠玮
\item 预计执行时长:5min
\item 执行状态:完成
\item 执行结果:通过
\item 问题单标识:---
\item 测试人员:史柠玮
\item 测试时间: \today
\item 未执行原因:---
\end{itemize}

\subsubsection{客户端-升级战}
\begin{itemize}
\item 测试用例名称:客户端-升级战
\item 标识:TEST-1.4
\item 测试用例综述:测试升级战是否可以成功执行并返回相应结果
\item 用例初始化:登录test用户
\item 前提和约束:已经拥有三只初始精灵
\item 测试步骤
  \begin{enumerate}
  \item
    \begin{itemize}
    \item 操作:主界面中选中一只Legendary精灵,点击Battle按钮,对手等级调到3,稀有度保持Common不变,点击Upgrade Tour按钮,等待战斗结束
    \item 结果和评估标准:提示战斗胜利,并且提示获得经验值,精灵升级
    \item 实测结果:与预期结果一致
    \end{itemize}
  \item
    \begin{itemize}
    \item 操作:主界面中选中一只Legendary精灵,点击Battle按钮,对手等级调到4,稀有度保持Common不变,点击Upgrade Tour按钮,立即点击skip按钮
    \item 结果和评估标准:提示战斗胜利,并且提示获得经验值,精灵升级
    \item 实测结果:与预期结果一致
    \end{itemize}

  \item
    \begin{itemize}
    \item 操作:主界面中选中一只Common精灵,点击Battle按钮,对手等级调到15,稀有度调整为Legendary,点击Upgrade Tour按钮,立即点击skip按钮
    \item 结果和评估标准:提示战斗失败
    \item 实测结果:与预期结果一致
    \end{itemize}



  \end{enumerate}
  
\item 测试用例终止条件:本测试用例的全部测试步骤被执行或因某种原因导致测试步骤无法执行(异常终止)。
\item 测试用例通过准则:本测试用例的全部测试步骤都通过即标志本用例为 “通过”。
\item 设计人员:史柠玮
\item 预计执行时长:10min
\item 执行状态:完成
\item 执行结果:通过
\item 问题单标识:---
\item 测试人员:史柠玮
\item 测试时间: \today
\item 未执行原因:---
\end{itemize}

\subsubsection{客户端-决斗战}
\begin{itemize}
\item 测试用例名称:客户端-决斗战
\item 标识:TEST-1.5
\item 测试用例综述:xxx
\item 用例初始化:xxx
\item 前提和约束:xxx
\item 测试步骤
  \begin{enumerate}
  \item
    \begin{itemize}
    \item 操作:主界面中选中一只Legendary精灵,点击Battle按钮,对手等级调到2,稀有度调整为Epic,点Duel Race按钮,立即点击skip按钮
    \item 结果和评估标准:提示战斗胜利,并且提示获得经验值,弹出命名窗口,输入new\_pm,返回主界面出现新精灵new\_pm,等级为2,稀有度为Epic
    \item 实测结果:与预期结果一致
    \end{itemize}

  \item
    \begin{itemize}
    \item 操作:主界面中选中一只Common精灵,点击Battle按钮,对手等级调到15,稀有度调整为Legendary,点击Duel Race按钮,立即点击skip按钮
    \item 结果和评估标准:提示战斗失败,弹出送出窗口,打开下拉框选择new\_pm,返回主界面精灵new\_pm消失
    \item 实测结果:与预期结果一致
    \end{itemize}

  \end{enumerate}
\item 测试用例终止条件:本测试用例的全部测试步骤被执行或因某种原因导致测试步骤无法执行(异常终止)。
\item 测试用例通过准则:本测试用例的全部测试步骤都通过即标志本用例为 “通过”。
\item 设计人员:史柠玮
\item 预计执行时长:5min
\item 执行状态:完成
\item 执行结果:通过
\item 问题单标识:---
\item 测试人员:史柠玮
\item 测试时间: \today
\item 未执行原因:---
\end{itemize}

\subsection{服务端功能测试(TEST-2)}

\subsubsection{服务端-查看用户登录登出情况}
\begin{itemize}
\item 测试用例名称:服务端-查看用户登录登出情况
\item 标识:TEST-2.1
\item 测试用例综述:检验服务端监视用户登录登出的正确性。
\item 用例初始化:已经注册过几个用户(test1, test2),将服务端打开
\item 前提和约束:已经注册过几个用户(test1, test2)
\item 测试步骤
  \begin{enumerate}
  \item
    \begin{itemize}
    \item 操作:登录test1用户
    \item 结果和评估标准:服务器端显示test1用户登录,并显示其IP地址以及接收端口
    \item 实测结果:与预期结果一致
    \end{itemize}
  \item
    \begin{itemize}
    \item 操作:登录test2用户
    \item 结果和评估标准:服务器端显示test2用户登录,并显示其IP地址以及接收端口
    \item 实测结果:与预期结果一致
    \end{itemize}
  \item
    \begin{itemize}
    \item 操作:将test1用户登出
    \item 结果和评估标准:服务器端显示test1用户离线,并且其IP地址以及接收端口变为 $-$
    \item 实测结果:与预期结果一致
    \end{itemize}

  \item
    \begin{itemize}
    \item 操作:将test1用户登出
    \item 结果和评估标准:服务器端显示test1用户离线,并且其IP地址以及接收端口变为 $-$
    \item 实测结果:与预期结果一致
    \end{itemize}
  \end{enumerate}
\item 测试用例终止条件:本测试用例的全部测试步骤被执行或因某种原因导致测试步骤无法执行(异常终止)。
\item 测试用例通过准则:本测试用例的全部测试步骤都通过即标志本用例为 “通过”。
\item 设计人员:史柠玮
\item 预计执行时长:5min
\item 执行状态:完成
\item 执行结果:通过
\item 问题单标识:---
\item 测试人员:史柠玮
\item 测试时间: \today
\item 未执行原因:---
\end{itemize}

\subsubsection{服务端-查看所有用户及其精灵}
\begin{itemize}
\item 测试用例名称:服务端-查看所有用户及其精灵
\item 标识:TEST-2.1
\item 测试用例综述:检验服务端监视用户及其精灵的正确性。
\item 用例初始化:已经注册过用户(test),将服务端打开
\item 前提和约束:已经注册过用户(test)
\item 测试步骤
  \begin{enumerate}
  \item
    \begin{itemize}
    \item 操作:点击服务端上的test用户
    \item 结果和评估标准:服务器端显示test用户拥有的精灵信息
    \item 实测结果:与预期结果一致
    \end{itemize}
  \item
    \begin{itemize}
    \item 操作:test用户进行一场决斗战,点击服务端上的test用户
    \item 结果和评估标准:服务器端显示test用户的精灵信息有相应的改变
    \item 实测结果:与预期结果一致
    \end{itemize}
  \end{enumerate}
\item 测试用例终止条件:本测试用例的全部测试步骤被执行或因某种原因导致测试步骤无法执行(异常终止)。
\item 测试用例通过准则:本测试用例的全部测试步骤都通过即标志本用例为 “通过”。
\item 设计人员:史柠玮
\item 预计执行时长:5min
\item 执行状态:完成
\item 执行结果:通过
\item 问题单标识:---
\item 测试人员:史柠玮
\item 测试时间: \today
\item 未执行原因:---
\end{itemize}


\end{document}
